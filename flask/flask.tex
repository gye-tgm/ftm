\documentclass[../main/main.tex]{subfiles}

\begin{document}

\section{Flask}  

In the following section we will see how we can quickly set up a fully
functional Flask project with the speed that the developers are
promising.

We assume that the required Python modules are already globally
installed, which is generally considered as bad practice, however,
without further ado, let's get started.

\subsection{Fundamental Setup}

\subsubsection{Project Structure}

In this section we will vaguely explain how the project should be
roughly structured.

\begin{lstlisting}
ftmatura/
  app/  [see below]
  test/ [see below]
  app.db
  config.py
  run.py
\end{lstlisting}

The \lstinline|test/| directory should actually be very obvious,
however, it's important to note that the test files all should begin
with the same prefix, such as \lstinline|test_|, so that we can run
all tests that begin with \lstinline|test_| with one simple matching
pattern \lstinline|test_*|: 

\begin{lstlisting}
test/
  test_base.py
  test_user.py
  test_login.py
  ...
\end{lstlisting}

In \lstinline|config.py| we will insert, as the name suggests, the
configuration of the application. 

\lstinline|run.py| is basically a script that will start the
application with or without the debug mode, depending on whether we are using
it for the production environment. 

The code that powers the software should be located inside the \lstinline|app/|
directory:

\begin{lstlisting}
app/
  models/
  models/__init__.py
  models/user.py
  models/train.py
  views/__init__.py
  views/user.py
  views/train.py
  templates/ [see below]
  static/ 
  app.py
  extensions.py
  config.py
  data.py
  ...
\end{lstlisting}

In \lstinline|templates/| we will put the code from the \textit{view}
(from MVC) component. These are basically HTML files that are fueled by
the very \textbf{powerful} Jinja 2 plugin.

In this example we use the so called \textit{functional structure} that
is presented in \cite{exploreflask:functional}. As the author assumes
that the modules are not fully independent from each other, he suggests
using the \textit{functional structure} during the FT Matura (instead of
using the divisional structure; more information under
\cite{exploreflask:functional}). 

In the \lstinline|template/| directory we will also setup a new directory for
only this page; so in the end, the template directory can take the following
structure: 

\begin{lstlisting}
template/
  user/
  train/
  errors/403.html
  errors/404.html
  base.html
  index.html
\end{lstlisting}

This structure is intended for middle sized projects, such as the one that will
be given in the FT Matura. 

\subsubsection{\lstinline|run.py|}

The \lstinline|run.py| is, essentially, the script file that will be executed to
start the webservice. Here, we will merely call the app with some debug flags
on or off. 

\begin{lstlisting}[caption=run.py, label=lst:run.py]
#!/usr/bin/env python
from app import create_app
from app.config import BaseConfiguration
app = create_app(BaseConfiguration)
#app.run(debug=True, use_reloader=False, use_debugger=False)
app.run(debug=True, host='0.0.0.0')
\end{lstlisting}

The one line that is commented out is vital for debugging with the
IDE, especially for PyCharm, as it turns off some additional features
that thwart the IDE from debugging with the \lstinline|pdb|. 

The file has a shebang on the first line, which means the first Python
interpreter that is found in the \lstinline|PATH| gets called. 

The file must also be given the execution rights via \lstinline|chmod u+x run.py|. 

\subsubsection{\lstinline|app/extensions.py|}

In this file we will declare the basic extensions that will be
initalized later on. 


\begin{lstlisting}
from flask.ext.sqlalchemy import SQLAlchemy
db = SQLAlchemy()

from flask.ext.login import LoginManager
login_manager = LoginManager()

from flask.ext.admin import Admin
admin = Admin()

import tweepy
twitter_api = tweepy.API()
\end{lstlisting}

\subsubsection{\lstinline|app/app.py|}

This file should essentially offer a \lstinline|create_app| function for
the user. This app depends on the \lstinline|config_object| that is
passed to the method. It will configure all the possible things such as
the extensions, the blueprints, the error handlers, and so on. The
following code snippet should give an vague overview of the vital
elements. 

There are some \lstinline|if app.config['SOME_FLAG']| checks that are
essential for testing. Sometimes, these are things that should not not
be configured during testing. 

\begin{lstlisting}
# imports omitted here ... 
def create_app(config_object):
    app = Flask(__name__)

    configure_app(app, config_object)

    if app.config['USE_HOOK']:
        configure_hook(app)
    configure_blueprints(app)
    configure_extensions(app)
    configure_error_handlers(app)

    if app.config['USE_ADMIN_INTERFACE']:
        configure_admin(app)
    return app


def configure_app(app, config_object):
    app.config.from_object(config_object)


def configure_admin(app):
    class AuthenticatedModelView(ModelView):
        def is_accessible(self):
            return login.current_user.is_authenticated() and login.current_user.role == ROLE_ADMIN

    admin.add_view(AuthenticatedModelView(User, db.session, endpoint='myuser'))
    admin.add_view(AuthenticatedModelView(Follow, db.session))
    admin.add_view(AuthenticatedModelView(SocialMediaAccount, db.session))
    admin.add_view(AuthenticatedModelView(FacebookAccount, db.session))
    admin.add_view(AuthenticatedModelView(TwitterAccount, db.session))
    admin.add_view(AuthenticatedModelView(Post, db.session))
    admin.add_view(AuthenticatedModelView(TwitterPost, db.session))
    admin.init_app(app)


def configure_extensions(app):
    # http://stackoverflow.com/questions/19437883/when-scattering-flask-models-runtimeerror-application-not-registered-on-db-w
    db.app = app
    db.init_app(app)

    login_manager.login_view = 'user.login'

    @login_manager.user_loader
    def load_user(id):
        return User.get(int(id))
    login_manager.init_app(app)

    Bootstrap(app)

    twitter_api.auth = tweepy.OAuthHandler(app.config['CONSUMER_KEY'],
                                           app.config['CONSUMER_SECRET'])
    twitter_api.auth.set_access_token(app.config['ACCESS_TOKEN'],
                                      app.config['ACCESS_TOKEN_SECRET'])


def configure_hook(app):
    twcr = TwitterCrawler()

    def run_schedule():
        while 1:
            schedule.run_pending()
            time.sleep(1)

    @app.before_request
    def before_request():
        # we shouldn't query so often...
        # https://dev.twitter.com/rest/public/rate-limiting
        schedule.every(1).minute.do(twcr.crawl_all)
        t = Thread(target=run_schedule)
        t.start()


def configure_blueprints(app):
    app.register_blueprint(track_mod, url_prefix='/track')
    app.register_blueprint(user_mod, url_prefix='/user')


def configure_error_handlers(app):
    @app.errorhandler(403)
    def forbidden_page(error):
        return render_template("errors/forbidden_page.html"), 403

    @app.errorhandler(404)
    def page_not_found(error):
        return render_template("errors/page_not_found.html"), 404

    @app.errorhandler(500)
    def server_error_page(error):
        return render_template("errors/server_error.html"), 500
\end{lstlisting}

\subsubsection{\lstinline|app/config.py|}
\label{sec:config.py}

The powerful and flexible \lstinline|config.py| file is one of Flask's
biggest asset:

\begin{lstlisting}[caption=app/config.py, label=lst:config.py]
from os.path import join, dirname, abspath
_cwd = dirname(abspath(__file__))
class BaseConfiguration(object):
    SECRET_KEY = 'a-new-secret-key-you-will-never-guess'
    SQLALCHEMY_DATABASE_URI = 'sqlite:///' + join(_cwd, 'app.db')
    SQLALCHEMY_ECHO = True

    USE_ADMIN_INTERFACE = True
    BOOTSTRAP_SERVE_LOCAL = True
    USE_HOOK = True


class TestConfiguration(BaseConfiguration):
    TESTING = True
    WTF_CSRF_ENABLED = False
    USE_ADMIN_INTERFACE = False
    USE_HOOK = False
    SQLALCHEMY_ECHO = False
    DATABASE = join(_cwd, 'test.db')
    SQLALCHEMY_DATABASE_URI = 'sqlite:///' + DATABASE
    LIVESERVER_PORT = 5001
\end{lstlisting}

These are some configuration variables that are important for an
appropriate setup of the SQLite database that we are going to use
throughout the exam. 

The SQLite database is also extraordinairly powerful, as it can be
used in the memory, as we can see in the \lstinline|TestConfiguration|
class. However, we will use a file as a database, since many issues can
occur due to Flask. 

Configuring from an BaseConfiguration object is simple and can be seen in the
following lines (or in \lstinline|app/app.py|):

\begin{lstlisting}
from app.config import BaseConfiguration
app = Flask(__name__)
app.config.from_object(BaseConfiguration)
\end{lstlisting}

\subsection{Blueprints}

Blueprints are essentially sets of operations that will be executed,
if they are registered into the \lstinline|app|. Here, we will use a
couple of them, as they increase the degree of modularity and
flexibility. To be more specific, we will use for each new page a
single blueprint. 

The registration will be done in the \lstinline|app/__init__.py|
file. Note that we can also specify the \lstinline|url_prefix|: 

\begin{lstlisting}
from app.users.views import mod as user_mod
app.register_blueprint(user_mod, url_prefix='/home')
\end{lstlisting}

While the \lstinline|app/users/views.py| could have the following
content:

\begin{lstlisting}
from flask import Blueprint, render_tmplate

mod = Blueprint('user', __name__, template_folder='templates')
@mod.route('/')
def index():
    return render_template('user/index.html')
\end{lstlisting}

To sum it up a little bit: for every page, we create a single module
inside the \lstinline|app/| module, and create a corresponding
\lstinline|views.py| file, in which we will declare the blueprints. 

Note that we can register the blueprints under several names, and not
just only a single one. 

While developing, it might happen that the \lstinline|url_prefix|
should be changed, due to an arbitrary reason, then we merely change
the \lstinline|url_prefix|, instead of every
\lstinline|@mod.route('/path/to/this/bp')| to something else, which is
indeed very convenient.

The first parameter in the \lstinline|Blueprint| constructor just
states the names under which the blueprint can be accessed. Should the
name be \lstinline|user|, then we can for instance use the handy
\lstinline|url_for('user.index')| function to get the corresponding
link to the \lstinline|index()| function of this blueprint. 

However, we must take care that there are no names collisions, which
means that we can't just register multiple non-identical blueprints
under the same name. 

\subsection{Example Views}

Now we will see how the blueprint is used in a real view file. This view file
contains the basic routes of the \lstinline|user| page, which contains the
functionalities such as logging in, logging out, and so on.

In \lstinline|render_template| we must pass all the variables that will be
accessed in the corresponding Jinja2 template file. 

\begin{lstlisting}[caption=app/views/user.py, label=lst:data.py]
from app.forms.user import RegisterForm, LoginForm, AddAccForm, AddFbForm, AddTwitterForm
from app.models import TwitterAccount, FacebookAccount, Follow
from flask.ext import login as flask_login

from flask import Blueprint, flash, redirect, url_for, render_template, request
from flask.ext.login import login_user, login_required, logout_user

from app.models import User

mod = Blueprint('user', __name__, template_folder='templates')

@mod.route('/')
def index():
    return render_template('user/index.html')


@mod.route('/login/', methods=('GET', 'POST'))
def login():
    form = LoginForm(request.form)
    if form.validate_on_submit():
        login_user(form.user)
        flash("Login successful", category="info")
        return redirect(request.args.get("next") or url_for('.index'))
    return render_template('user/login.html', form=form)


@mod.route('/logout/')
@login_required
def logout():
    logout_user()
    return redirect(url_for('.index'))


@mod.route('/register/', methods=("POST", "GET"))
def register():
    form = RegisterForm(request.form)
    if form.validate_on_submit():
        user = User.create(username=form.name.data,
                           password=form.password.data)
        flash("Registration successful", category="success")
        return redirect(url_for(".index"))
    return render_template('user/register.html', form=form)
\end{lstlisting}


\subsection{The Model Classes (+Relationships)}

\subsubsection{Simple Declaration}
In this section we want to briefly discuss how to implement a single
model class via the \lstinline|Flask-SQLAlchemy| module. 

We first declare a helper class that we will put into
\lstinline|app/data.py|: 

\begin{lstlisting}[caption=app/data.py, label=lst:data.py]
from app.extensions import db

class CRUDMixin(object):
    """
    Inheriting this class eliminates common code for the fundamental CRUD commands.
    """
    __table_args__ = {'extend_existing': True}

    id = db.Column(db.Integer, primary_key=True)

    @classmethod
    def create(cls, commit=True, **kwargs):
        instance = cls(**kwargs)
        return instance.save(commit=commit)

    @classmethod
    def get(cls, id):
        return cls.query.get(id)

    @classmethod
    def get_or_404(cls, id):
        return cls.query.get_or_404(id)

    def update(self, commit=True, **kwargs):
        for attr, value in kwargs.iteritems():
            setattr(self, attr, value)
        return commit and self.save() or self

    def save(self, commit=True):
        db.session.add(self)
        if commit:
            db.session.commit()
        return self

    def delete(self, commit=True):
        db.session.delete(self)
        return commit and db.session.commit()
\end{lstlisting}

Basically, it just reduces a few lines for each CRUD command, however,
on the long term, this will reduce the number of code lines, thus the
developing time, proportionally.

As we know from SQLAlchemy, every object that we need to insert into the
database has to be first added from the \lstinline|session| object, and
then the \lstinline|session| object has to commit the session. This
might get tedious, especially if a 5 hour timer is ticking in the
background.

Additionally, it contains an integer, called \lstinline|id|, as a
primary key, which happens to be the case in a vast majority of the
model classes.

Let's consider the implementation of the following \lstinline|User|
class: 

\begin{lstlisting}[caption=app/models/user.py, label=lst:user.py]
from app.data import CRUDMixin
from app.extensions import db
from app.models.account import SocialMediaAccount
from app.models.facebook.account import FacebookAccount
from app.models.post import Post
from app.models.twitter.account import TwitterAccount
from flask.ext.login import UserMixin


ROLE_USER = 0
ROLE_ADMIN = 1


class User(db.Model, UserMixin, CRUDMixin):
    __tablename__ = 'user'

    username = db.Column(db.String, unique=True)
    password = db.Column(db.String)
    role = db.Column(db.Integer, default=ROLE_USER)

    accounts = db.relationship(SocialMediaAccount, backref='user', lazy='dynamic')
    posts = db.relationship(Post, backref='user', lazy='dynamic')

    following = db.relationship('Follow', backref="follower", foreign_keys='Follow.a_id')

    def is_authenticated(self):
        return True

    def is_active(self):
        return True

    def is_anonymous(self):
        return False

    def get_id(self):
        return str(self.id)

    def get_following(self):
        return [f.b for f in self.following]

    def get_followers(self):
        return [f.follower for f in self.follow_assoc]

    def is_valid_password(self, password):
        # normally you would compare the hash here
        return self.password == password

    def get_twitter_accounts(self):
        return list(filter(lambda x: isinstance(x, TwitterAccount), self.accounts))

    def get_facebook_accounts(self):
        return list(filter(lambda x: isinstance(x, FacebookAccount), self.accounts))

    def __repr__(self):
        return "<User #{:d}>".format(self.id)
\end{lstlisting}

We do not consider \lstinline|UserMixin|, which is a class used for
authentication, as we will rather focus on the so famous
\lstinline|db.Model| and the \lstinline|CRUDMixin| class that we have
just declared.

There are some crucial facts that must be considered during the
implementation of the model classes:

\begin{itemize}
  \item \lstinline|__tablename__| is the tablename that we will also
    find in the database. Having this attribute is
    \textbf{obligatory}. 
  \item Normally, every table must also have a \textbf{primary key}, however,
    by inheriting from the convenient \lstinline|CRUDMixin| class, we
    also inherit a very common \lstinline|id| primary key. 
  \item Each model class that inherits from \lstinline|db.Model|
    is essentially just a table in the database. 
\end{itemize}

Details on how to have relationships such as inheritance, one-to-many,
many-to-many, etc., will be considered in Sections
\ref{sec:relationships}, \ref{sec:onetomany}, \ref{sec:manytomany},
\ref{sec:inheritance}. (Hint: CTRL+F might come handy) 

We will come back to this code later on, as there are some undiscussed
methods which are more useful for the authentication and authorization
part, as seen in section \ref{sec:auth}.

Nevertheless, the columns, as seen in the docs of the
\lstinline|Flas-SQLAlchemy| \cite{mitsuhiko:declare_models}, can have
the following datatypes: 

\begin{tabular}{ | l | l | l | } \\ \hline
Integer & an integer & \lstinline|db.Integer| \\ \hline
String (size) & a string with a maximum length &  \lstinline|db.String(42)|\\ \hline
Text & some longer unicode text &  \lstinline|db.Text|  \\ \hline
DateTime & date and time expressed as Python datetime object. & \lstinline|db.DateTime|\\ \hline
Float & stores floating point values & \lstinline|db.Float| \\ \hline
Boolean & stores a boolean value & \lstinline|db.Boolean| \\ \hline
Pickle & Typestores a pickled Python object &  \lstinline|db.Pickle| \\ \hline
LargeBinary & stores large arbitrary binary data &  \lstinline|db.LargeBinary|\\ \hline
\end{tabular}


\subsubsection{Relationships}
\label{sec:relationships}

Relationships are expressed with \lstinline|relationship()|. The
first paramater can either be the class or just the (string) name of the
class \cite{mitsuhiko:declare_models}, since one might want to declare
the class with the foreign key at an uncertain time in the future. 

Let's jump right into the more concrete associations in the following
sections. 

\subsubsection{One-To-Many}
\label{sec:onetomany}

If we consider the one single line from \ref{lst:user.py}:

\begin{lstlisting}
  posts = db.relationship(Post, backref='user', lazy='dynamic')  
\end{lstlisting}

With the assignment of the \lstinline|backref| variable, the
\lstinline|Post| class will also have a back reference to the user
class, thus \lstinline|Post| objects have an attribute \lstinline|user|.

The corresponding \lstinline|Post| class follows: 

\begin{lstlisting}[caption=app/models/post.py, label=post.py]
from app import db
from app.data import CRUDMixin

class Post(db.Model, CRUDMixin):
    __tablename__ = 'post'

    content = db.Column(db.String)
    time = db.Column(db.Date)

    user_id = db.Column(db.Integer, db.ForeignKey('user.id'))
\end{lstlisting}

Notice that the Post class, which is also the class on the
\textit{many} side, must have a foreign key of the corresponding user
class. Note the following line: 

\begin{lstlisting}
    user_id = db.Column(db.Integer, db.ForeignKey('user.id'))  
\end{lstlisting}

This declares the foreign key that is important, as we know from the
official SQLAlchemy tutorial. 

% todo: one-to-one

\subsubsection{Many-To-Many (Association Object)}
\label{sec:manytomany}

Many-to-many relationships are always joined together with an
association table. This table can be declared like any other table. 

% todo: better example

\begin{lstlisting}
from app import db
from app.data import CRUDMixin
from app.models import User

class Follow(db.Model, CRUDMixin):
    """
    a follows b
    """
    __tablename__ = 'follow'
    a_id = db.Column(db.Integer, db.ForeignKey('user.id'), primary_key=True)
    b_id = db.Column(db.Integer, db.ForeignKey('user.id'), primary_key=True)

    following_since = db.Column(db.Date, nullable=False, default=db.func.now())

    # issue: http://stackoverflow.com/questions/18807322/sqlalchemy-foreign-key-relationship-attributes
    a = db.relationship(User, foreign_keys="Follow.a_id")
    b = db.relationship(User, foreign_keys="Follow.b_id")
\end{lstlisting}

\subsubsection{Inheritance}
\label{sec:inheritance}

If a class needs to be inherited, then the super class must own a so called
\lstinline|discriminator| that can be of any type (typically string), as this is
used to discriminate (in a positive way) the different subclasses. 

We must register the discriminator into the \lstinline|__mapper_args__|. 

A social media account can be a FacebookAccount or a TwitterAccount. An example
will be provided below, as classes with names A, B, and C are the worst
examples to explain something: 

\begin{lstlisting}
from app.extensions import db
from app.data import CRUDMixin

class SocialMediaAccount(db.Model, CRUDMixin):
    __tablename__ = 'smaccount'
    user_id = db.Column(db.Integer, db.ForeignKey('user.id'))

    discriminator = db.Column(db.Integer)
    __mapper_args__ = {'polymorphic_on': discriminator}  
\end{lstlisting}

Now the Facebook and Twitter accounts respectively. In the subclasses, we also
need to give the discriminator a value by setting the
\lstinline|__mapper_args__['polymorphic_identity']| variable correspondingly:

\begin{lstlisting}
from app.extensions import db
from app.models import SocialMediaAccount

class FacebookAccount(SocialMediaAccount):
    __tablename__ = 'fbaccount'
    __mapper_args__ = {'polymorphic_identity': 'fb_account'}

    id = db.Column(db.Integer, db.ForeignKey('smaccount.id'), primary_key=True)
    url = db.Column(db.String)
\end{lstlisting}

\begin{lstlisting}
from app.extensions import db
from app.models import SocialMediaAccount

class TwitterAccount(SocialMediaAccount):
    __tablename__ = 'twitteraccount'
    __mapper_args__ = {'polymorphic_identity': 'tw_account'}

    id = db.Column(db.Integer, db.ForeignKey('smaccount.id'), primary_key=True)
    username = db.Column(db.String)  
\end{lstlisting}

\subsection{Authentication and Authorization (Login)}
\label{sec:auth}

In this section we will discuss how to setup a simple login system
with the \lstinline|Flask-Login| module. 

The basic code of the user can already be seen in Listing
\ref{lst:user.py}. This class inherits from \lstinline|UserMixin|,
which is a class that is provided and necessitated by the
\lstinline|Flask-Login| module. 

\subsection{Admin Pages (CRUD)}

An excellent module that generates an admin page under the URI
\lstinline|/admin| wherein the admin can perform CRUD operations and
other related operations. 

We will create a customized class that will derive the
\lstinline|ModelView|, however, this class also makes a small
authorization and authentication check. 

\begin{lstlisting}
admin = Admin()
admin.init_app(app)

class AuthenticatedModelView(ModelView):
  def is_accessible(self):
    return login.current_user.is_authenticated() and login.current_user.role == ROLE_ADMIN

admin.add_view(AuthenticatedModelView(User, db.session, endpoint='myuser'))

\end{lstlisting}

The \lstinline|endpoint| paramter indicates the suffix of the URI under which we
can access this particular page. For instance, in this code snippet, we can
access the user CRUD page with \lstinline|/admin/myuser|, otherwise it will use
the class name, and this might lead to unexpected results. 

In the background it also uses blueprints. 


\subsection{\lstinline|manage.py| optional}

This snippet can be used so that the manager of the system can have a handy
script to initalize the database, deploy the server, ... and so on.

\begin{lstlisting}
from flask.ext.script import Manager

from app import create_app
from app.models.user import ADMIN
from app.extensions import db

app = create_app()
manager = Manager(app)


@manager.command
def run():
    app.run()


@manager.command
def initdb():
    db.drop_all()
    db.create_all()

    # create an admin account 
    User.create(username='admin', password='letmein', role=ADMIN)

manager.add_option('-c', '--config', dest="config", required=False, help="config file")

if __name__ == "__main__":
    manager.run()
\end{lstlisting}

\subsection{Forms}

We use forms to get the input from the user. Generating them as HTML is
not a hard task. Just follow the following concept: 

\begin{lstlisting}
from flask.ext.wtf import Form
# and more imports ... 

class LoginForm(Form):
    name = StringField('Username', [InputRequired()])
    password = PasswordField('Password', [InputRequired()])
    submit = SubmitField('Login')

    # TODO: will validate_user be called before validate_password ?
    def validate_name(form, field):
        try:
            form.user = User.query.filter(User.username == form.name.data).one()
        except NoResultFound:
            raise ValidationError("User does not exist")

    def validate_password(form, field):
        if hasattr(form, 'user') and not form.user.is_valid_password(form.password.data):
            raise ValidationError("Invalid password")
\end{lstlisting}

In the Jinja2 template we merely have to call the following code snippet
to generate the full form: 

\begin{lstlisting}
{{ wtf.quick_form(form) }}
\end{lstlisting}

\subsection{Scheduling}

Using the \lstinline|schedule| module, we can perfectly create timed tasks that
will be executed in the background. 

Read the \lstinline|schedule| documentation for more options. However, it's very
important to note that these requests have to be started in
\lstinline|@app.before_request|, and it's important to set the
\lstinline|use_reloader=False| in \lstinline|run.py|, otherwise it will start
the schedulers twice. 

\begin{lstlisting}
def configure_hook(app):
    twcr = TwitterCrawler()

    def run_schedule():
        while 1:
            schedule.run_pending()
            time.sleep(1)

    @app.before_request
    def before_request():
        schedule.every(1).minute.do(twcr.crawl_all)
        t = Thread(target=run_schedule)
        t.start()
\end{lstlisting}

\subsection{Bootstrap GUI}

The bootstrap GUI can be applied if our base file, which we will definitely
declare, extends from the given "bootstrap/base.html" file. 

Jinja is actually very simple, but if something's wrong, the documentation is
also sufficient for our requirements.  

\subsubsection{Macros}

These are very important macros that can be used to render all kind of forms. It
also renders the error messages, should there be any. 

\begin{lstlisting}

    <div class="form_field">
        {{ field.label(class="label") }}
        
            
            {{ field(class=css_class, **kwargs) }}
            <ul class="errors">
                <li>{{ error|e }}</li></ul>
        
            {{ field(**kwargs) }}
        
    </div>



    <a href="{{ url_for(url) }}">{{ text }}</a>

\end{lstlisting}

\subsubsection{\lstinline|base.html|}

Every file in our projects needs a base, thus the file that is provided below
should be suffice. 

\begin{lstlisting}



 Choose a title! 


    <nav class="navbar navbar-default">
        <div class="container">
            <div class="navbar-header">
                <a class="navbar-brand" href="#">
                    Track-It
                </a>
            </div>

            <div class="navbar-left">
                <p class="navbar-text">
                    {{ render_link("track.index", "Home") }}
                </p>
                <p class="navbar-text">
                    <a href="{{ url_for("track.me") }}">Me</a>
                </p>
                
                    <p class="navbar-text">
                        <a href="/admin">Admin</a>
                    </p>
                
            </div>

            <div class="navbar-right">
                
                    <p class="navbar-text">
                        Signed in as {{ current_user.username }}
                    </p>
                    <p class="navbar-text">
                        {{ render_link("user.settings", "Settings") }}
                    </p>
                    <p class="navbar-text">
                        {{ render_link("user.logout", "Logout") }}
                    </p>

                
                    <p class="navbar-text">
                        {{ render_link("user.login", "Login") }}
                    </p>
                
            </div>
        </div>
    </nav>



<div class="container">
    
    {{ utils.flashed_messages() }}
</div>


\end{lstlisting}

\subsubsection{\lstinline|login.html|}

Since we have such a comprehensive \lstinline|base.html| file as a base, we can
just extend it and gain a page with the same layout as the parent. In the
following code snippet, we can see how quickly a login form can be generated via
WTForms. 

\begin{lstlisting}


    {{ super() }}
    
    <div class="container">
        <div class="row">
            <div class="col-xs-4">
                {{ wtf.quick_form(form) }}
            </div>
        </div>
    </div>

\end{lstlisting}

\subsection{Logging}

At the very beginning of the program, we must also set the settings of the
logging module.

\begin{lstlisting}
logging.basicConfig(filename='flask.log', level=logging.INFO)
\end{lstlisting}

Then we can use the \lstinline|app| to log the behavior of the program:

\begin{lstlisting}
app.logger.info/warning/error(message)
\end{lstlisting}

\subsection{Authorization}

This is a small, lightweight and compact way to handle user authorizations. This
is a code snippet from the official Flask code snippets \cite{flask:auth}.

\begin{quote} This snippet is pretty simple. You just need to replace
  get_current_user_role() with however you get the user's current role and
  error_response() with however you want to notify the user that they are not
  logged in. After you do that, you should be good to go \cite{flask:auth}.
\end{quote}

\begin{lstlisting}
def requires_roles(*roles):
    def wrapper(f):
        @wraps(f)
        def wrapped(*args, **kwargs):
            if get_current_user_role() not in roles:
                return error_response()
            return f(*args, **kwargs)
        return wrapped
    return wrapper
\end{lstlisting}

\begin{quote}
Usage is equally as simple as the snippet itself. This is just a decorator that
you pass the required roles into. The required roles can be any type of object,
not just strings. Do note that if you use a login extension such as Flask-Login,
you should call it after the login_required (or equivalent) decorator
\cite{flask:auth}.
\end{quote}

\begin{lstlisting}
@app.route('/user')
@required_roles('admin', 'user')
def user_page(self):
    return "You've got permission to access this page."
\end{lstlisting}


% \bibliographystyle{ieeetr}
% \ifcsdef{mainfile}{}{\bibliography{../primary}}

\end{document}
