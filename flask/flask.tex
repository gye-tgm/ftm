\documentclass[../main/main.tex]{subfiles}

\begin{document}

\section{Flask}  

In the following section we will see how we can quickly set up a fully
functional Flask project with the speed that the developers are
promising.

We assume that the required Python modules are already globally
installed, which is generally considered as bad practice, however,
without further ado, let's get started.

\subsection{Fundamental Setup}

\subsubsection{Project Structure}

In this section we will vaguely explain how the project should be
roughly structured.

\begin{lstlisting}
ftmatura/
  app/  [see below]
  tests/ [see below]
  app.db
  config.py
  run.py
\end{lstlisting}

The \lstinline|test/| directory should actually be very obvious,
however, it's important to note that the test files all should begin
with the same prefix, such as \lstinline|test_|, so that we can run
all tests that begin with \lstinline|test_| with one simple matching
pattern:

\begin{lstlisting}
test/
  test_base.py
  test_user.py
  test_login.py
  ...
\end{lstlisting}

In \lstinline|config.py| we will insert, as the name suggests, the
configuration of the application. 

\lstinline|run.py| is basically a script that will start the
application with or without the debug mode, depending on whether we
are using it for the production environment. 

The code that powers the software should be located inside the app/
directory:

\begin{lstlisting}
app/
  models/
  models/__init__.py
  models/model1.py
  models/model2.py
  templates/ [see below]
  static/ 
  page1/
  page2/ 
  utility1.py
  utility2.py
\end{lstlisting}

In \lstinline|templates/| we will put the code from the \textit{view}
(from MVC) component. These are basically HTML files that are fueled
by the Jinja 2 plugin, a very \textbf{powerful} plugin. 

For each subpage $i$ we will create a \lstinline|page|$_i$ module that
will contain some basic Python modules, such as \lstinline|views.py|,
\lstinline|forms.py|; files that contain the logic of the page, which
will be discussed later on. In the \lstinline|template/| directory we
will also setup a new directory for only this page; so in the end, the
template directory can take the following structure: 

\begin{lstlisting}
template/
  page1/
  page2/
  base.html
  index.html
\end{lstlisting}

\begin{lstlisting}
from os.path import join, dirname, abspath

_cwd = dirname(abspath(__file__))

class BaseConfiguration(object):
    SECRET_KEY = this-should-be-secret'
    SQLALCHEMY_DATABASE_URI = 'sqlite:///' + join(_cwd, 'app.db')
    SQLALCHEMY_ECHO = True
\end{lstlisting}




\end{document}